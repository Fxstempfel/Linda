\documentclass[10pt, a4paper]{article}

\usepackage[utf8]{inputenc}
\usepackage[T1]{fontenc}
\usepackage[french]{babel}
\usepackage{graphicx}
\usepackage{wrapfig}
\usepackage{titlesec} 
\usepackage[top=3.2cm, bottom=3.2cm, left=3.2cm, right=3.2cm]{geometry}
\usepackage{amsmath} %maths
\usepackage{amsfonts} %maths
\usepackage{fancyhdr} %footer
\usepackage[scaled=0.95]{helvet}
\usepackage{lmodern} %font
\usepackage[font=sf,labelfont=sf]{caption} 
\usepackage[titles]{tocloft} %toc
\usepackage[hidelinks]{hyperref} %refs cliquables
\usepackage{color}
\usepackage{xcolor}
\usepackage{xstring}

\title{\bl{\hrule}\vspace{1em}\Huge\textbf{Projet Systèmes Concurrents \\Rapport Final Étape 1}\\[-0.7em]\bl{\hrule}}
\author{\Large\textsc{\bl{B}oissin} \bl{T}hibaut \and \Large\textsc{\bl{S}tempfel} \bl{F}rançois-Xavier}
\date{3 janvier 2016}


%%%%%%%%%%%%%%%%%%%%%%%%%%
% TOC
	% Change of name
	\addto\captionsfrench{
	  \renewcommand{\contentsname}%
	    {Sommaire}%
	}
	
	% Change of subsec font
	\renewcommand{\cftsubsecfont}{\sffamily}
	
	% Change of subsec page numbers font
	\renewcommand{\cftsubsecpagefont}{\sffamily}
	
	\renewcommand{\cftsubsubsecpagefont}{\sffamily}
	
	% In case of indent pbs, a \cftsetindents command exists
	% To remove a page number, \cftpagenumbersoff (command)
%%%%%%%%%%%%%%%%%%%%%%%%%%%

% Change of numerotation & indent
\renewcommand{\thesection}{\Roman{section}}
\renewcommand{\thesubsection}{\bl{$\bullet$}}

\titleformat{\section}[block]{\bfseries\Large}{\thesection.}{.5em}{}[\vspace{-0.7em}\bl{\titlerule}]
\titleformat{\subsection}[block]{\bfseries\large}{\thesubsection}{10 pt}{}[\vspace{-0.7em}\bl{\titlerule}]
\titleformat{\subsubsection}[block]{\bfseries\large}{\thesubsubsection.}{8 pt}{}
\titlespacing{\subsection}{12 pt}{*4}{*3}
\titlespacing{\subsubsection}{25 pt}{*3}{*1.5}

\definecolor{myBlue}{RGB}{4,38,204}
\newcommand{\bl}[1]{\textcolor{myBlue}{#1}}

%\renewcommand{\section}[1]{\section{\bl{\StrChar{#1}{1}}\StrDel{\StrChar{#1}{1}}}}

% Change of figure name
\addto\captionsfrench{
	\renewcommand{\figurename}{\sffamily \textsc{Figure}}
}

%%%%%%%%%%%%%%%%%%%%%%%%%%%
% Change small caps font
% phv for helvetica
% sc for small caps
\DeclareTextFontCommand{\textsc}{\fontfamily{phv}\fontshape{sc}\selectfont} 
%%%%%%%%%%%%%%%%%%%%%%%%%%%

%\setlength{\parindent}{0pt}


%%%%%%%%%%%%%%%%%%%%%%%%%%%
\renewcommand{\sectionmark}[1]{\markboth{#1}{}}
% Page layouts
\pagestyle{fancy}
\lhead{\sffamily\leftmark}
\chead{}
\rhead{\sffamily \emph{}}
\lfoot{}
\cfoot{\sffamily\thepage}
\rfoot{}
\renewcommand{\headrulewidth}{0pt}
\renewcommand{\footrulewidth}{0pt}
%%%%%%%%%%%%%%%%%%%%%%%%%%%

%%%%%%%%%%%%%%%%%%%%%%%%%%%
% To wrap a figure :

%	\begin{wrapfigure}[nb of rows]{r}{width}
%		\centering
%		\includegraphics[]{}
%		\caption{}
%		\label{}
%	\end{wrapfigure}
	
%%%%%%%%%%%%%%%%%%%%%%%%%%%




\begin{document}

\sffamily

\maketitle


\vspace{2em}

\section*{Structures de données et stratégies utilisées}

	Pour gérer le caractère bloquant des méthodes \texttt{read} et \texttt{take}, nous avons choisi d'utiliser deux \texttt{Map}, contenant respectivement les \texttt{read} et les \texttt{take} en attente. Les clés sont les tuples correspondant au motif recherché et les valeurs sont des \texttt{Queue} (implémentées par des \texttt{LinkedList}) qui contiennent les \textit{callback} associés aux requêtes.
	
	L'espace de tuples est lui représenté par une \texttt{List}.
	
	La gestion de la concurrence se fait grâce à des verrous, par le biais de blocs \texttt{synchronized} et éventuellement des méthodes \texttt{wait} et \texttt{notify}.
	
	À noter que nous avons opté pour 	une solution qui permet de factoriser le code : dans les méthodes \texttt{read} et \texttt{take}, nous utilisons la méthode \texttt{eventRegister} avec les bons paramètres.
	
\section*{Description des méthodes sensibles}

	\subsection{write}
		Au sein du bloc verrouillé, on cherche d'abord si des \texttt{read}, puis des \texttt{take}, sont en attente d'un tuple correspondant à celui concerné. Si c'est le cas :
		\begin{itemize}
			\item pour les \texttt{read}, on appelle tous les \textit{callbacks} correspondants,
			\item pour les \texttt{take}, on appelle le premier \textit{callback} et on le retire de la file (s'il existe).
		\end{itemize}
		Si l'on a pas trouvé de \texttt{take} correspondant, on ajoute le tuple à l'espace de tuples.
		
	\subsection{take et read}
		Ces deux méthodes sont quasiment les mêmes : on fait appel à la fonction \texttt{eventRegister} avec les bons paramètres et on appelle la fonction \texttt{wait} dans le bloc \texttt{synchronized}. Le \textit{callback} passé en paramètre est une instance de la classe interne \texttt{BlockingCallback} implémentant l'interface \texttt{Callback}. Le \texttt{call} de cette classe permet en fait de stocker le tuple dans l'attribut \texttt{result} et donc de le récupérer, puis de réveiller le processus en attente (ayant exécuté un \texttt{wait}) avec \texttt{notify.}
		
		Seule différence : si le tuple est présent avant l'exécution de \texttt{read}, il ne faut pas exécuter le \texttt{wait}.
		
	\subsection{eventRegister}
		Si le timing est \texttt{IMMEDIATE}, on exécute la méthode \texttt{tryRead} ou \texttt{tryTake} selon le cas. Si un tuple correspondant n'a pas été trouvé, on ajoute le tuple et son \textit{callback} aux opérations en attente. Sinon, on sort du bloc \texttt{synchronized} et on exécute le \textit{callback}.
		
		Si le timing est \texttt{FUTURE}, on ajoute simplement le tuple et son \textit{callback} à la liste des opérations en attente.
		
\section*{Remarques du second binôme}

	Nous avons pris en compte les trois propositions d'amélioration du second binôme. La première était de remplacer les \texttt{if} des méthodes \texttt{read} et \texttt{take} par des \texttt{while}, ce que nous avons fait. Cela permet d'apporter plus de robustesse au programme, puisqu'il semble que le \texttt{wait} dans un bloc \texttt{synchronized} puisse être activé sans que le \texttt{notify} n'ait été appelé. 
	
	La deuxième remarque mettait en lumière le manque de commentaires, et nous avons rectifié le tir en en rajoutant là où il nous semblait en manquer, de façon à permettre une lecture plus facile et rapide de notre code.
	
	Enfin, nous n'avons pas corrigé la différence entre les méthodes \texttt{readAll} et \texttt{takeAll}. En effet, nous avions une bonne raison de ne pas les coder de la même façon et en particulier de ne pas utiliser le \texttt{tryRead} dans le \texttt{readAll} comme on utilise le \texttt{tryTake} dans le \texttt{takeAll} : le \texttt{take} retire le tuple de l'espace de tuple, celui-ci n'est donc plus considéré par la suite. Ce n'est pas le cas pour le \texttt{read} et un code du \texttt{readAll} symétrique à celui du \texttt{takeAll} impliquerait potentiellement (s'il y a au moins un tuple correspondant dans l'espace) une boucle infinie. La méthode la plus simple que nous ayons trouvée pour écrire le \texttt{readAll} reste donc notre première solution.
	
\end{document}

